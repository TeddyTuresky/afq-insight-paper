\section*{Abstract}

The white matter contains long-range connections between different brain
regions. Tractometry uses diffusion-weighted magnetic resonance imaging (dMRI)
data to quantify tissue properties along the trajectories of these connections
in the human brain in vivo\cite{yeatman2012tract}. In previous work, the results
of tractometry were usually analyzed using mass univariate approaches: group
comparisons or regression models computed separately for each point along every
one of the tracts.  Alternatively, tissue properties such as fractional
anisotropy (FA) and mean diffusivity (MD) were computed for a specific tract of
interest based on an a priori hypothesis. In the present work, we developed a
method based on the sparse group lasso\cite{simon2013sparse} that takes into
account tissue properties measured in all of the tracts, and selects informative
features by enforcing sparsity, both at the level of individual tracts and
tissue properties, but also across the entire set of tracts and all of the
measured tissue properties. The sparsity penalties for each of these constraints
is identified using a nested cross-validation scheme. Using data from a previous
study that measured dMRI in patients with amyotrophic lateral sclerosis (ALS)
and matched controls\cite{sarica2017corticospinal}, we demonstrate that this
method is accurate, exceeding the previous results, that were based on a priori
feature selection, in classifying patients and controls, with ~84\% accuracy, an
area under the receiver operating characteristic curve of 0.93, and an average
precision of 0.95. Moreover, our method automatically identifies as important
for this classification the parts of the white matter known to be affected by
ALS within the corticospinal tract. In a regression setting, data from another
previous study\cite{yeatman2014lifespan} can be used to accurately predict
``brain age''. Thus, this multivariate analysis approach both (a) achieves high
cross-validated accuracy for precision medicine applications of dMRI data and
(b) identifies relevant features of brain anatomy to further our neuroscientific
understanding of clinical disorders.
