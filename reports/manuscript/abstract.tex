\section*{Abstract}

The white matter contains long-range connections between different
brain regions. Tractometry uses diffusion-weighted magnetic resonance
imaging (dMRI) data to quantify tissue properties (e.g. fractional
anisotropy (FA), mean diffusivity (MD), etc.), along the trajectories
of these connections\cite{yeatman2012tract}. Statistical inference from
tractometry data usually either (a) averages these quantities along
the length of the tract in each individual, or (b) performs analysis
point-by-point along each tract, with group comparisons or regression
models computed separately for each point along every one of the tracts.
In the present work, we developed a method based on the sparse group
lasso (SGL) \cite{simon2013sparse} that takes into account tissue
properties measured along all of the tracts, and selects informative
features by enforcing sparsity, not only at the level of individual
tracts and tissue properties, but also across the entire set of tracts
and all of the measured tissue properties. The sparsity penalties for
each of these constraints is identified using a nested cross-validation
scheme that guards against over-fitting and simultaneously identifies
the correct level of sparsity. We demonstrate the accuracy of the method
in two settings: in a regression setting, dMRI is used to accurately
predict ``brain age'' \cite{yeatman2014lifespan, Brown2012-so}. In this
case, the weights are distributed throughout the white matter indicating
that many different regions of the white matter change with development
and contribute to the prediction of age. In a classification setting,
patients with amyotrophic lateral sclerosis (ALS) are accurately
distinguished from matched controls\cite{sarica2017corticospinal} and
SGL automatically identifies FA in the corticospinal tract as important
for this classification -- correctly finding the parts of the white
matter known to be affected by the disease. Thus, SGL makes it possible
to leverage the multivariate relationship between diffusion properties
measured along multiple tracts to make accurate predictions of subject
characteristics while simultaneously discovering the most relevant
features of the white matter for the characteristic of interest.
