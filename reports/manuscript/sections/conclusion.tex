\section*{Conclusion}

We present here a novel method for analysis of dMRI tractometry data that relies
on the Sparse Group Lasso \cite{simon2013sparse} to (a) make accurate
predictions of phenotypic properties of individual subjects while,
simultaneously, (b) identifying the most important features of the white matter
in a completely data-driven approach. Importantly, through the use of a nested
cross-validation procedure, the method adapts to the constraints of small
datasets while also leveraging the power of larger datasets. Moreover, the
method is broadly applicable to a wide range of research questions: it performs
well in predicting both continuous variables, such as biological age, as well as
categorical variables, such as whether a person is a patient or a healthy control.
In both of these cases, SGL out-performs previous algorithms that have been
developed for these tasks \cite{sarica2017corticospinal, Richard2018-ux}. The
nested cross-validation approach used to fit the model and make both predictions
and inferences from it guards against overfitting and tunes the degree of
sparseness required by the algorithm. This means that both phenomena that are
locally focused in a particular anatomical location and a particular property of
diffusion (e.g., FA in the CST), as well as widely distributed phenomena can be
accurately captured.

Specifically, we demonstrated that the algorithm correctly detects the fact that
ALS, which is a disease of lower motor neurons, is localized to the
cortico-spinal tract. This recapitulates the results of previous analysis of
these same data, using a targetted ROI-based approach
\cite{sarica2017corticospinal}. In contrast, in analysis of biological age, the
coefficients identified by the algorithm are very widely distributed across many
parts of the white matter, mirroring previous results with this dataset that
show a large and continuous distribution of life-span changes in white matter
properties \cite{yeatman2014lifespan}.

Taken together, these results demonstrate the promise of the group-regularized
regression approach. Even at the scale of dozens of subjects, the results
provided by SGL are both accurate and interpretable. Thus, this multivariate
analysis approach both (a) achieves high cross-validated accuracy for precision
medicine applications of dMRI data and (b) identifies relevant features of brain
anatomy that can further our neuroscientific understanding of clinical
disorders.

Neuroscience has entered an era in which consortium efforts are putting together
large datasets of high-quality dMRI data to address a variety of scientific
questions \cite{jernigan2016ping, jernigan2018abcd, alexander2017open,
Miller2016-hw, VanEssen2012}. The volume and complexity of these data pose a
substantial challenge. Dimensionality reduction with tractometry, followed by
analysis with the approach we present here promises to capitalize on the wealth
of data and on the co-measurement of interesting and important phenotypical data
about brain health and about the participants' cognitive abilities. We also
expect the group-regularized approach to improve with larger datasets.

SGL has many other potential applications in neuroscience, because of the
hierarchical and grouped nature of many of the data -- that are collected in
multiple sample points within anatomically-defined areas. For example, this
method may be useful to understand the relationship between fMRI recordings and
behavior, where activity in each voxel may co-vary with voxels within the same
anatomical region and form features and groups of features. Similarly,
large-scale multi-electrode recordings of neural activity in awake behaving
animals are becoming increasingly feasible \cite{steinmetz2018distributed,
Jun2017-gv} and these recordings can form features (neurons) and groups (neurons
within an anatomical region). More ambitiously perhaps, this approach may be
used to understand the role of correlations in so-called resting-state fMRI
time-series and behavior, where pairwise correlations between voxels in
different anatomical regions are features in the matrix and features may be
grouped by pairs of anatomical regions. Given the large number of voxels in the
surface of the gray matter, and given that correlations increase the number of
features by a factor of $n^2$ this would be a challenging problem to solve using
SGL.

The results we present here also motivate extensions of the method using more
sophisticated cost functions. For example, the fused sparse group lasso (FSGL)
\cite{zhou2012} extends SGL to enforce additional spatial structure: smoothness
in the variation of diffusion metrics along the tracts. As brain measurements
include additional structure (for example, bilateral symmetry), future work
could also incorporate overlapping group membership for each entry in the tract
profiles \cite{Rao2014-xm}. For example, a measurement could come from the
corpus callosum, but also from the right hemipshere. This would also require
extending the cost function used here to incorporate these constraints.
Similarly, unsupervised dimensionality reduction \cite{Chamberland2019-mu} could
also benefit from constraints based on grouping.

The method is packaged as open-source software called AFQ-Insight that is openly
available, and provides a clear API to allow for extensions of the method. The
sofware integrates within a broader automated fiber quantification software
ecosystem: AFQ \cite{yeatman2012tract}, which extracts tractometry data from raw
and processed dMRI datasets, as well as AFQ-Browser, which visualizes
tractometry data and facilitates sharing of the results of dMRI studies
\cite{yeatman2018browser}. To facilitate reproducibility and ease use of the
software, the results presented in this paper are also provided in
\url{https://github.com/richford/afq-insight-paper} as a series of Jupyter
notebooks
\cite{kluyver2016jupyter}

