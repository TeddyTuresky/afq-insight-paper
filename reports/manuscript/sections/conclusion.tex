\section*{Conclusion}

We present here a novel method for analysis of dMRI tractometry data. This
method relies on the Sparse Group Lasso \cite{simon2013sparse} to provide both
acurate prediction of the phenotypic properties of individual subjects based on
their dMRI data, but also provides interpretable results by identifying the
features that are important across subjects to make these predictions. The
method is broadly applicable: it performs well in predicting both continuous
variables, such as biological age, as well as discrete variable, such as whether
a person is a patient or a healthy control. In both of these cases, SGL
out-performs previous algorithms that have been developed for these tasks. In
addition, the nested cross-validation approach that we have developed tunes the
degree of sparseness required by the algorithm, such that both very local
phenomena, as well as widely distributed phenomena are accurately captured.

Specifically, we demonstrated that the algorithm correctly detects the fact that
 ALS, which is a disease of lower motor neurons, is localized to the
 cortico-spinal tract. This recapitulates the results of previous analysis of
 these same data, using a targetted ROI-based approach. In contrast, in analysis
 of biological age, the coefficients identified by the algorithm are very widely
 distributed.

Taken together, these results demonstrate the promise of a group regularized
regression approach. Even at the scale of dozens of subjects, the results
provided by SGL are both accurate and interpretable. We expect that additional
data will improve the performance of these algorithms. Neuroscience has entered
an era in which large consortium efforts are putting together large datasets
that can be analyzed using this approach. Future work will apply this method to
these datasets. For example, to the Human Connectome Project dataset
\cite{VanEssen2012}

These results also motivate extensions of the method using more sophisticated
cost functions. For example, the fused sparse group lasso (FSGL)
\cite{zhou2012} extends SGL to enforce additional spatial structure: smoothness
in the variation of diffusion metrics along the tracts. As brain measurements
include additional structure (for example, bilateral symmetry), future work
could also incorporate overlapping group membership for each entry in the tract
profiles. For example, a measurement could come from the corpus callosum, but
also from the right hemipshere. This would also require extending the cost
function used here to incorporate these constraints.

The method is packaged as open-source software called AFQ-Insight that is openly
available. The sofware integrates within a broader automated fiber
quantification software ecosystem: AFQ \cite{yeatman2012tract}, which extracts
tractometry data from raw dMRI, as well as AFQ-Browser, which visualizes
tractometry data and facilitates sharing of the results of dMRI studies
\cite{yeatman2018browser}.

\begin{itemize}
  \item Feature selection + prediction
  \item Part of the AFQ ecosystem
  \item Open source, reproducibility
\end{itemize}
