\section*{Introduction}

\begin{itemize}

Diffusion-weighted Magnetic Resonance Imaging (dMRI) provides a unique view into
the physical properties of the connections that comprise the brain white matter
\cite{wandell2016clarifying}. Among the benefits of the method are the fact that
it is non-invasive, and feasible to conduct in clinical scanners, and in wide
variety of populations: healthy adults, but also clinical populations, as well
as young or elderly subjects. While the measurements are usually conducted with
voxels at the millimeter scale, within the diffusion time usually used, water
molecules within each voxel can probe the structure of white matter at the
micrometer scale, providing important information about the physical structure
of the white matter, such as the distribution of fiber orientations within each
voxel \cite{Frank2002-iz, Frank2001-xf}. With more sophisticated measurements
that include several diffusion-weightings, more information can be extracted
about the composition of different tissue type in the voxel
\cite{Jeurissen2014-ab, Kaden2016-ho}. In addition, phenomenological
descriptions of the diffusion properties, such as its mean diffusivity or
fractional anisotropy can be derived from dMRI measurements
\cite{Pierpaoli1996-vj}. Some of these descriptions are ambiguous in terms of
their interpretation, as variance in these properties can relate to several
different underlying physical causes. For example, fractional anisotropy can
decrease both because of demyelination, but also because of decreased density,
or because of partial volume with other tissue types or between crossing fiber
populations \cite{Beaulieu1996-fn, Beaulieu2002-tl}. Even so, these descriptions
have proven useful in characterizing individual differences or differences
between populations \cite{}.

To relate the diffusion in each voxel to the underlying macro-structure of
long-range connections between different brain regions, methods for
computational tract-tracing from diffusion MRI, or tractography combine the
estimates of fiber orientations in each voxel to form streamlines that traverse
the volume of the white matter \cite{Conturo1999-je, Mori2002-qi}. These methods
have been under increased scrutiny and several lines of investigation have
raised critiques of their validity \cite{Maier-Hein2017-vb, Thomas2014-ki}. On
the other hand, there have been efforts to shore up the inferences made with
these methods \cite{Pestilli2014NatMeth, Takemura2016-sh, Smith2013-nc,
Smith2015-cx, Smith2015-zt, Rheault2018-wk}. Importantly, though discovering
novel fiber bundles requires extraordinary evidence, and delineating the exact
location of the termination of the streamlines in the gray matter, there is
little dispute that tractography can accurately delineate major white matter
bundles that are known to exist within the core of the white matter
\ariel{citation needed}.

Using this fact, one of the most powerful methods currently available to put
macro- and micro-structure together is \emph{tractometry} (also known as
tract-based morphometry): assessment of the physical properties of the white
matter along specific tracts \cite{Bells2011-cf}. Though there are several
different available implementations of this overall idea, the principles are
similar \cite{yeatman2012, Yendiki2011-ay, Wassermann2016-iv, ODonnell2009-uu}:
tractometry begins by delineating the parts of the white matter that belong to
different major tracts and axonal fiber bundles, such as the corticospinal tract
or arcuate fasciculus, and samples the biophysical properties (such as
fractional anisotropy or mean diffusivity) along the length of these bundles. In
some cases, tissue properties along the length of each bundle are summarized by
taking the mean along each bundle, but there is a large body of evidence showing
that there is systematic variability within each bundle. This justifies
retaining the individual samples along the length of each fiber
\cite{yeatman2012, colby2012, ODonnell2009-uu}. However, while this retains
important information about each individual's white matter, it also presents
statistical challenges due to the dimensionality of the data. Based on
tractometry, researchers may choose to compare different individuals to each
other. This is usually done following one of the following approaches:

\begin{enumerate}
\item Mass univariate approaches: In this approach comparisons between groups or
across individuals are done at each node of each tract, for each one of the
diffusion metrics available at that point. This approach is exhaustive, but the
challenge with this approach is that the are multiple points to compare, and
statistical power is compromised by a multiple comparison problem. Different
approaches can be taken to resolving this challenge. For example, Colby and
colleagues \cite{colby2012} used a non-parametric resmapling approach to correct
for family-wise error across the universe of different comparisons available in
the mass univariate approach \cite{Nichols2002-zu, Nichols2003-yy}.
\item ROI-based approach: An alternative is using an ROI based approach. In this
approach, the multiple comparison problem is circumvented by selecting just a
few tracts to compare in each individual, or even focusing on particular
segments of these tracts. This approach is very powerful when the biological
basis of the process of interest is relatively well understood (for a recent
example, see \cite{huber2018rapid}).
\item ROI-based selection, followed by multivariate analysis. Here, an ROI is
selected based on \emph{a priori} knowledge, and all the nodes or voxels in the
ROI are used together to fit a model that can predict differences between
individuals.
\end{enumerate}

In the present work, we present a novel framework for analysis of tractometry
that simultaneously selects the features for analysis, and fits a model to these
features. Rather than a comparative approach, this framework takes a predictive
approach to analysis \cite{Yarkoni2017-bi}. Using this approach, we need to
first deal with the large and asymmetric dimensionality of the data: tractometry
data usually has many more features (i.e., number of measurements per
individual) than samples (number of subjects). This regime is the target of
several statistical learning techniques, and is often solved by various forms of
regularization. For example, Tikhonov regularization shrinks the solution such that
each feature. The Lasso. One of the enabling

