\section*{Introduction}

Diffusion-weighted Magnetic Resonance Imaging (dMRI) provides a unique
view into the physical properties of the connections that comprise
the brain white matter. While the measurements are usually conducted
with voxels at the millimeter scale, water molecules within each voxel
diffuse with characteristic lengths at the micrometer scale, providing
aggregate information about the physical structure of the white matter,
including the density of axons and distribution of fiber
orientations within each voxel \cite{wandell2016clarifying}. Even though
metrics derived from diffusion measurements are ambiguous in terms
of their underlying biological interpretation \cite{Jones2013-xv},
analyzing the variance in these properties has proven useful in
characterizing individual differences in cognitive function,
characterizing differences between populations and detecting brain
abnormalities associated with disease \cite{Thomason2011-qn}.

To relate the diffusion in each voxel to the macro-structure of
long-range connections between different brain regions, methods for
computational tract-tracing from diffusion MRI, or tractography, combine
the estimates of fiber orientations in each voxel to form streamlines
that traverse the volume of the white matter \cite{Conturo1999-je,
Mori2002-qi}. These methods have been under increased scrutiny and
several lines of investigation have raised critiques of their validity
\cite{Maier-Hein2017-vb, Thomas2014-ki}. On the other hand, there
have been efforts to shore up the inferences made with these methods
\cite{Pestilli2014NatMeth, Takemura2016-sh, Smith2013-nc, Smith2015-cx,
Smith2015-zt, Rheault2018-wk}. Importantly, though discovering novel
tracts requires extraordinary evidence, and delineating
the exact cortical termination of the streamlines in the gray matter
is still prone to error, there is little dispute that tractography
can accurately define the location of several major white matter
tracts that are known to exist within the core of the white matter
\cite{Maier-Hein2017-vb, Catani2002-vu}.

Leveraging this fact, one of the most powerful methods currently
available to put macro- and micro-structure together is
\emph{tractometry}: assessment of the physical properties of the white
matter along specific tracts \cite{Bells2011-cf}. Though there are
several different available implementations of this overall idea,
the principles are similar \cite{yeatman2012tract, Yendiki2011-ay,
Wassermann2016-iv, ODonnell2009-uu}: tractometry begins by delineating
the parts of the white matter that belong to different major tracts
(i.e. anatomical or functional groups of white matter fibers),
such as the corticospinal tract or arcuate fasciculus, assigning
tractography generated streamlines to ``bundles,'' which approximate
the anatomical tracts, and sampling biophysical properties (such as
fractional anisotropy or mean diffusivity) along the length of these
bundles. Whereas the identification of the anatomical correlates
of tractography-generated streamlines is an open problem, the
correspondence between tractometry's bundles and anatomical tracts is
well supported \cite{Catani2002-vu}. Tract-based diffusion metrics are
therefore appropriate degrees of freedom with which to understand the
white-matter correlates of cognitive function or disease.

In some previous tractometry-based studies, tissue properties along
the length of each tract were summarized by taking the mean along
each bundle, but there is a large body of evidence showing that there
is systematic variability in the values of diffusion metrics along the trajectory of each bundle. This
justifies retaining the individual samples along the length of each
bundle \cite{yeatman2012tract, colby2012, ODonnell2009-uu}. While this
retains important information about each individual's white matter, it
also presents statistical challenges due to the dimensionality of the
data. Based on tractometry, researchers may choose to compare different
individuals to each other. This is usually done according to one of the
following approaches:

\begin{enumerate}

\item Mass univariate approaches: In this approach comparisons between
groups or across individuals are done independently at each node
of each tract, for each one of the diffusion metrics available at
that point. This approach is exhaustive, but statistical power is
compromised by a multiple comparison problem. Different approaches
can be taken to resolving this challenge. For example, Colby and
colleagues \cite{colby2012} used a non-parametric resmapling approach to
correct for family-wise error across the different possible comparisons
\cite{Nichols2002-zu, Nichols2003-yy}.

\item Region of interest(ROI)-based approaches: An alternative that
circumvents the multiple comparison problem is to select just a few
tracts to compare in each individual, or even focusing on particular
segments of these tracts based on \emph{a priori} hypotheses. This
approach is very powerful when the biological basis of the process
of interest is relatively well understood (for a recent example, see
\cite{huber2018rapid}).

\item ROI-based selection, followed by multivariate analysis: Here,
an ROI is selected based on \emph{a priori} knowledge, and all the
nodes or voxels in the ROI are used together to fit a model that can
predict differences between individuals. An example of that is the
``profilometry'' framework, in which different diffusion metrics
from a single tract are combined together to provide input to a
multivariate analysis of covariance, and linear discriminant analysis
\cite{dayan2016profilometry}.

\end{enumerate}

Generally speaking, analysis methods should balance predictive
accuracy with descriptive power \cite{Murdoch2019-ax, Breiman2001-uz}.
Accordingly, tractometry analysis should simultaneously capitalize on
all the data across all tracts to make the best possible prediction,
while also retaining and elucidating spatial information about the
locations that are most informative for a prediction. In the present
work, we developed a novel framework for analysis of tractometry that
simultaneously selects the features for analysis, and fits a model
to these features. We use a linear modeling approach, which aims to
predict phenotypical variance in a group of subjects, based on a linear
combination of the features estimated with tractometry.

Using this approach, we first need to deal with the large and
asymmetric dimensionality of the data: tractometry data usually has
many more features (i.e., number of measurements per individual) than
samples (number of subjects), which makes inferences from the data
about phenotypical differences between individuals ill-posed. This
regime is the target of several statistical learning techniques, and
is often solved by various forms of regularization. For example,
Tikhonov regularization shrinks the solution such that the sum of
squared contributions from the individual features are minimized
\cite{Hoerl2000-ij}. Another solution to the problem is provided by
the Lasso algorithm, which instead minimizes the sum of the absolute
values of contributions of each feature \cite{Tibshirani1996-qs}. This
tends to shrink to zero the contributions of many of the features,
providing results that are both accurate and interpretable. When
additional structure is available in the organization of the data,
regularization algorithms can take advantage of this structure. For
example, if the features lend themselves to a natural division into
different groups, the group lasso (GL) can be used to select groups
of features, rather than individual features \cite{Yuan2006-ky}.
The Sparse Group Lasso (SGL) elaborates on this idea by providing
control both of group sparsity, as well as overall sparsity of the
solutions \cite{simon2013sgl}. Because the features measured with
tractomery lend themselves to grouping based on the tracts from which
each measurement is taken, GL and SGL could provide a useful tool
for linear model fitting in problems of this form. Here we, first,
develop an implementation of SGL that is well suited to the analysis of
tractometry data and, second, demonstrate the power and flexibility of
this approach by applying it to both classification (disease diagnosis)
and continuous prediction (age) problems from previously published
studies \cite{sarica2017corticospinal, yeatman2014lifespan}.
